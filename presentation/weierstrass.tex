\section{Approximationssatz von Weierstrass}

\frame{\frametitle{Approximationssatz von Weierstrass}
	\pause
	\begin{satz}
		Die Menge der Polynome liegt dicht in $C^0([a,b],\mathbb{R})$.
	\end{satz}
	\pause
	%\vspace{\baselineskip}
	Mit anderen Worten:
	\pause
	Für jedes $f\in C^0$ und jedes $\epsilon>0$ gibt es ein Polynom $p(x)$, sodass:
	\begin{equation*}
		|f(x) - p(x)| < \epsilon
	\end{equation*}
	für alle $x\in[a,b]$.
	
	\pause
	\begin{block}{Bemerkung}
		Für den Beweis kann man oBdA annehmen, dass $[a,b] = [0,1]$. Ansonsten skaliert man $f$ einfach.
	\end{block}
}

\frame{\frametitle{Beweis 1}
	\pause
	Betrachten das n-te $Bernstein-Polynom:$
	\pause
	\begin{equation*}
		p_n(x) = \sum_{k=0}^n\binom{n}{k} c_kx^k(1-x)^{n-k},
	\end{equation*}
	wobei $c_k = f(\frac{k}{n})$. \\
	\ \\
	\pause
	Behauptung: $p_n(x)$ konvergiert gleichmäßig gegen $f$ für $n\to\infty$. \\
	\pause
	Dazu definieren wir:
	\begin{equation*}
		r_k(x) = \binom{n}{k} x^k(1-x)^{n-k} \pause\ge 0
	\end{equation*}
}

\frame{
	Für $r_k(x)$ gelten folgende Identitäten: \pause
	\begin{equation*}
		\sum_{k=0}^nr_k(x) = 1, \qquad \pause \sum_{k=0}^n(k-nx)^2r_k(x) = nx(1-x)
	\end{equation*}
	\pause
	Beweis kommt später. \pause
	 \\
	Damit gilt: \pause
	\begin{equation*}
		p_n(x) = \sum_{k=0}^nc_kr_k(x), \pause \qquad f(x)\pause =f(x)\sum_{k=0}^nr_k(x) 
		\pause =\sum_{k=0}^nf(x)r_k(x)
	\end{equation*}
	\pause

	$\Rightarrow p_n(x) - f(x)\pause =\sum_{k=0}^n(c_k-f(x))r_k(x)$
}

\frame{
	Sei $\epsilon>0$. \pause Wollen zeigen: \pause $|p_n(x) - f(x)| <\epsilon \quad \forall x\in[0,1]$ und n groß genug. \\
	\pause
	Da $f$ gleichmäßig stetig auf $[0,1]$, existiert $\delta>0$, sodass:\pause
	\begin{align*}
		|t-s| <\delta \Rightarrow |f(t) - f(s)| < \frac{\epsilon}{2}
	\end{align*}
	\pause
	Definieren: \pause $K_1 = \left\{k\in \left\{0,...,n\right\}: |\frac{k}{n}-x|<\delta\right\}\quad$ \pause 
	und $\quad K_2 = \left\{0,...,n\right\}\setminus K_1.$\pause
	\begin{equation*}
		\begin{aligned}
			|p_n(x) - f(x)| \pause = \Big|\sum_{k=0}^n(c_k)-f(x))r_k(x)\Big| \pause\le\sum_{k=0}^n|c_k-f(x)|r_k(x) \\
			\pause = \sum_{k\in K_1}|c_k-f(x)|r_k(x) + \sum_{k\in K_2}|c_k-f(x)|r_k(x)
		\end{aligned}
	\end{equation*}
	\pause
	Werden beide Summen gegen $\frac{\epsilon}{2}$ abschätzen.
} 

\frame{
	\colorbox{yellow}{$\sum_{k\in K_1}|c_k-f(x)|r_k(x)$}\\
	\pause
	\ \\
	Es gilt: $|t-s| <\delta \Rightarrow |f(t) - f(s)| < \frac{\epsilon}{2}$ \pause
	\begin{equation*}
		\Rightarrow|c_k-f(x)| \pause = \Big|f\Big(\frac{k}{n}\Big)-f(x)\Big| \pause <\frac{\epsilon}{2} 
		\qquad\forall k\in K_1, \pause\qquad\Big(\Big|\frac{k}{n}-x\Big|<\delta \quad \forall k\in K_1\Big)
	\end{equation*}
	\pause
	Mit $\sum_{k=0}^nr_k(x) =1$ folgt: \pause
	\begin{equation*}
		\sum_{k\in K_1}|c_k-f(x)|r_k(x)\pause\le\sum_{k\in K_1}\frac{\epsilon}{2}r_k(x) 
		\pause\le\sum_{k=0}^n\frac{\epsilon}{2}r_k(x) \pause = \frac{\epsilon}{2}
	\end{equation*}
}

\frame{
	\colorbox{yellow}{$\sum_{k\in K_2}|c_k-f(x)|r_k(x)$}\\
	\ \\
	\pause
	Für $k\in K_2$ gilt:
	\begin{equation*}
		\Big|\frac{k}{n}-x\Big| \ge \delta \quad\pause\Leftrightarrow\quad 
		|k-nx|\ge n\delta \quad\pause\Leftrightarrow\quad (k-nx)^2\ge(n\delta)^2
	\end{equation*}
	\pause
	Mit der zweiten Identität von $r_k$ folgt: 
	\pause
	\begin{equation*}
		nx(1-x) = \sum_{k=0}^n(k-nx)^2r_k(x)\pause\ge \sum_{k\in K_2}(k-nx)^2r_k(x)
		\pause\ge \sum_{k\in K_2}(n\delta)^2r_k(x)
	\end{equation*}
	\pause
	Also:
	\begin{equation*}
		\sum_{k\in K_2}r_k(x) \pause\le \frac{nx(1-x)}{(n\delta)^2}\pause = \frac{x(1-x)}{n\delta^2}
	\end{equation*}
}

\frame{
	\colorbox{yellow}{$\sum_{k\in K_2}|c_k-f(x)|r_k(x)$}\\
	\ \\
	Es gilt: $x(1-x)\le\frac{1}{4} \quad\forall x\in[0,1].$ \pause Damit erhält man:
	\begin{equation*}
		\sum_{k\in K_2}r_k(x) \le \frac{x(1-x)}{n\delta^2}\pause\le\frac{1}{4n\delta^2}
	\end{equation*}
	\pause
	$|f|$ nimmt als stetige Funktion Maximum $M\in\mathbb{R}$ auf $[0,1]$ an.\pause
	\begin{equation*}
		\Rightarrow |c_k-f(x)|\pause\le 2M \quad\pause\forall x\in [0,1]
	\end{equation*}
	\pause
	Mit beiden Anschätzungen folgt für n groß genug:
	\pause
	\begin{equation*}
		\sum_{k\in K_2}|c_k-f(x)|r_k(x) \pause\le 2M\sum_{k\in K_2}r_k(x) 
		\pause\le \frac{M}{2n\delta^2} \pause\le \frac{\epsilon}{2}
	\end{equation*}
}

\frame{
	Bleiben noch die beiden Identitäten von $r_k$ zu zeigen. \\
	\pause
	Erinnerung:
	\begin{equation*}
		r_k(x) = \binom{n}{k}x^k(1-x)^{n-k}
	\end{equation*}
	\pause
	Betrachten binomischen Lehrsatz: \pause$(x+y)^n = \sum_{k=0}^n\binom{n}{k}x^ky^{n-k}$ 
	\pause und setzen $y = 1-x$.
	\pause
	\begin{equation*}
		\Rightarrow \sum_{k=0}^nr_k(x) \pause = (x+(1-x))^n  \pause = 1
	\end{equation*}
	\pause
	Dies war gerade die erste Identität.
}

\frame{
	Zweite Identität: $\sum_{k=0}^n(k-nx)^2r_k(x) = nx(1-x).$ \\
	\pause
	Betrachten wieder binomischen Lehrsatz und leiten zweimal nach $x$ ab:
	\pause
	\begin{equation}
		n(x+y)^{n-1} = \sum_{k=0}^n \binom{n}{k}kx^{k-1}y^{n-k}
	\end{equation}
	\pause
	\begin{equation}
		n(n-1)(x+y)^{n-2} = \sum_{k=0}^n\binom{n}{k}k(k-1)x^{k-2}y^{n-k}
	\end{equation}
	Setzen wieder $y=1-x$. \pause Multiplizieren (1) mit $x$ und (2) mit $x^2$:
	\pause
	\begin{equation}
		nx \pause = \sum_{k=0}^n\binom{n}{k}kx^k(1-x)^{n-k} \pause = \sum_{k=0}^nkr_k(x)
	\end{equation}
	\pause
	\begin{equation}
		n(n-1)x^2 \pause = \sum_{k=0}^n\binom{n}{k}k(k-1)x^k(1-x)^{n-k} \pause = \sum_{k=0}^nk(k-1)r_k(x)
	\end{equation}
}

\frame{
	$nx = \sum_{k=0}^nkr_k(x) \quad (3), \qquad n(n-1)x^2 = \sum_{k=0}^nk(k-1)r_k(x) \quad(4)$ \\
	\ \\
	\pause
	Umstellen von (4) und einsetzen von (3):
	\begin{equation}
		\sum_{k=0}^nk^2r_k(x) \pause = n(n-1)x^2 + \sum_{k=0}^nkr_k(x) \pause = n(n-1)x^2 + nx
	\end{equation}
	\pause
	Aus (3), (5) und der ersten Identität folgt:
	\begin{equation*}
		\begin{aligned}
			\sum_{k=0}^n(k-nx)^2r_k(x) \pause = \sum_{k=0}^nk^2r_k(x) - 2nx\sum_{k=0}^nkr_k(x) 
			+(nx)^2\sum_{k=0}^nr_k(x) \\
			\pause = n(n-1)x^2 + nx \pause - 2(nx)^2 \pause + (nx)^2 = \pause -nx^2 \pause + nx \pause 
			= nx(1-x)
		\end{aligned}
	\end{equation*}
	\pause
	$\Rightarrow$ Zweite Identität\qed
}

\frame{\frametitle{Beweis 2}
	\pause
	Können Problem umformulieren: \\
	\pause
	Sei $g(x) = f(x) - (mx + b), \quad$ mit $m= f(1)-f(0)$ und $b = f(0)$ \\
	\pause
	$\Rightarrow g(0) = g(1) = 0$ \\
	\pause
	$\Rightarrow$ $g$ durch Polynome approximierbar $\Leftrightarrow$ f durch Polynome approximierbar\\
	\ \\
	\pause
	Sei also o.B.d.A $f(0) = f(1) = 0$ \pause und $f(x) = 0 \quad\forall x\in\mathbb{R}\setminus[0,1].$\\
	\pause
	$\Rightarrow$ $f$ stetig auf ganz $\mathbb{R}$\\
	\ \\
	\pause
	Betrachten Hilfsfunktion:
	\pause
	\begin{equation*}
		\beta_n(t) = b_n(1-t^2)^n\pause\ge0\quad -1\le t\le1,
	\end{equation*}
	\pause
	wobei $b_n$ so gewählt, dass $\int_{-1}^1\beta_n(t)\,dt = 1.$		
}

\frame{
	Definieren:
	\begin{equation*}
		P_n(x) := \int_{-1}^1f(x+t)\beta_n(t)\,dt \quad x\in[0,1]
	\end{equation*}
	\pause
	Behauptung: $P_n$ ist Polynom und konvergiert gleichmäßig gegen $f$.\\
	\pause
	Substituieren mit $u = x+t$:
	\pause
	\begin{equation*}
		\Rightarrow P_n(x) = \int_{x-1}^{x+1}f(u)\beta_n(u-x)\,du \pause = \int_0^1f(u)\beta_n(u-x)\,du
	\end{equation*}
	da $f=0$ außerhalb von $[0,1]$\\
	\pause
	$\beta_n$ ist Polynom $\pause\Rightarrow$ können Potenzen von $x$ mit Linearität aus Integral ziehen \\ 
	\pause $\Rightarrow$ $P_n(x)$ ist Polynom. 
}

\frame{
	Zeigen zuerst:\\
	\pause
	Sei $\delta>0 \Rightarrow\beta_n(t)$ konvergiert gleichmäßig gegen $0$ \pause für $\delta\le|t|\le1.$\\
	\ \\
	\pause
	Erinnerung: $\beta_n(t) = b_n(1-t^2)^n$.\\
	\pause
	Es gilt:
	\begin{equation*}
		1 = \int_{-1}^1\beta_n(t)\,dt \pause\ge \int_{\frac{-1}{\sqrt{n}}}^{\frac{1}{\sqrt{n}}}b_n(1-t^2)^n\,dt
		\pause\ge \int_{\frac{-1}{\sqrt{n}}}^{\frac{1}{\sqrt{n}}}b_n(1-\frac{1}{n})^n\,dt 
		\pause= b_n\frac{2}{\sqrt{n}}(1-\frac{1}{n})^n
	\end{equation*}
	\pause
	Da $(1-\frac{1}{n})^n$ beschränkt, gilt:
	\pause
	\begin{equation*}
		b_n\le(1-\frac{1}{n})^{-n}\frac{\sqrt{n}}{2}\pause\le c\sqrt{n}
	\end{equation*}	
	für eine Konstante $c\in \mathbb{R}$ und alle $n\ge2$.
}

\frame{
	$b_n\le c\sqrt{n}$\\
	\pause
	Daraus folgt für $\delta\le|t|\le1$:
	\pause
	\begin{equation*}
		\beta_n(t) = b_n(1-t^2)^n\pause\le c\sqrt{n}(1-t^2)^n\pause\le c\sqrt{n}(1-\delta^2)^n
		\pause\to 0\qquad (n\to\infty)
	\end{equation*}
	\pause
	$\Rightarrow \beta_n\to 0$ gleichmäßig auf $\delta\le|t|\le1$. \\
	\pause \ \\
	Zeigen nun: $P_n\to f$ gleichm\"aßig.\\
}

\frame{
	Sei $\epsilon>0$. \pause Da $f$ gleichm\"aßig stetig auf $[0,1]$ existiert $\delta>0$, sodass:
	\begin{equation*}
		|t|<\delta\Rightarrow |f(x+t) - f(x)|<\frac{\epsilon}{2}.
	\end{equation*}
	\pause
	Erinnerung: $\int_{-1}^1\beta_n(t)\,dt = 1$
	\pause 
	\begin{equation*}
		\begin{aligned}|P_n(x) - f(x)| \pause= \bigg|\int_{-1}^1f(x+t)\beta_n(t)\,dt -f(x)\bigg| \\
			\pause
			= \bigg|\int_{-1}^1(f(x+t)-f(x))\beta_n(t)\,dt\bigg| \pause\le\int_{-1}^1|f(x+t)-f(x)|\beta_n(t)\,dt \\ 
			\pause
			= \int_{|t|<\delta}|f(x+t)-f(x)|\beta_n(t)\,dt + \int_{|t|\ge\delta}|f(x+t)-f(x)|\beta_n(t)\,dt \\
			\pause
			\le \frac{\epsilon}{2} \pause+ \int_{|t|\ge\delta}|f(x+t)-f(x)|\beta_n(t)\,dt
		\end{aligned}
	\end{equation*}
}	

\frame{
	\colorbox{yellow}{$\int_{|t|\ge\delta}|f(x+t)-f(x)|\beta_n(t)\,dt$}\\
	\ \\
	\pause
	Sei $M\in\mathbb{R}$ das Maximum von $|f|$ auf $[0,1]$.\\
	\pause
	Wissen: $\beta_n(t)$ konvergiert gleichmäßig gegen $0$ für $\delta\le|t|\le1$.
	\pause
	\begin{equation*}
		\Rightarrow \int_{|t|\ge\delta}|f(x+t)-f(x)|\beta_n(t)\,dt \pause\le 2M\int_{|t|\ge\delta}\beta_n(t)\,dt
		\pause\le \frac{\epsilon}{2}
	\end{equation*}
	für n groß genug.
	\qed
}