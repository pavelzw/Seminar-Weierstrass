\section{Approximationssatz von Weierstrass}

\frame{\frametitle{Approximationssatz von Weierstrass}
	\pause
	\begin{block}{Satz 1}
		Die Menge der Polynome liegt dicht in $C^0([a,b],\mathbb{R})$.
	\end{block}
	\pause
	%\vspace{\baselineskip}
	Mit anderen Worten:
	Für jedes $f\in C^0$ und jedes $\epsilon>0$ gibt es ein Polynom $p(x)$, sodass:
	\begin{equation*}
		|f(x) - p(x)| < \epsilon
	\end{equation*}
	für alle $x\in[a,b]$.
	
	\pause
	\begin{block}{Bemerkung}
		Für den Beweis kann man oBdA annehmen, dass $[a,b] = [0,1]$. Ansonten skalliert man $f$ einfach.
	\end{block}

	
		
	
}
	