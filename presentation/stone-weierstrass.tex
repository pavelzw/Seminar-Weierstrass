\section{Approximationssatz von Stone-Weierstraß}

\begin{frame}{Funktionenalgebra}
    \begin{defi}[Funktionenalgebra]
        Eine Menge \( \mathcal{A} \subset \mathcal{C}^0(M) \) heißt 
        \textit{Funktionenalgebra}, wenn sie geschlossen unter Addition, 
        skalarer Multiplikation und Funktionenmultiplikation ist, 
        d.~h. \( \forall f,g \in \A, c \in \mathbb{R} \)
        \[ f + g \in \mathcal{A}, \quad c \cdot f \in \mathcal{A}, \quad f \cdot g \in \mathcal{A}. \]
    \end{defi}
\end{frame}

\begin{frame}{Funktionenalgebra}
    \begin{defi}[Verschwinden]
        Eine Funktionenalgebra \(\A \subset \CM\) 
        \textit{verschwindet} an einem Punkt \(p \in M\), falls 
        \[ f(p) = 0 \;\forall f \in \mathcal{A}. \]
    \end{defi}
    \pause
    \begin{bsp}
        \( \A := \set{ x \mapsto \sum_{j=1}^n a_n x^n \;|\; n\in \N, a_j \in \R, j = 1,\ldots, n } \subset \mathcal{C}^0([-1,1]) \) 
        verschwindet im Punkt \(0\).
    \end{bsp}
\end{frame}

\begin{frame}{Punkte unterscheiden}
    \begin{defi}
        Eine Funktionenalgebra \(\A \subset \CM\) \textit{unterscheidet Punkte}, 
        wenn \( \forall p_1, p_2 \in M \) eine Funktion \(f \in \A\) existiert, 
        sodass \( f(p_1) \neq f(p_2) \).
    \end{defi}
    \pause
    \begin{bsp}
        \( \A := \set{ x \mapsto \sum_{j=0}^n a_{j} x^{2j} \;|\; n\in\N, a_j \in \R, j = 1,\ldots, n } 
        \subset \mathcal{C}^0([-1,1]) \)
        separiert die Punkte \( z \) und \(-z\) (\(z \in [-1,1] \setminus \set{0} \)) nicht, da \( \forall f \in \A \) gilt 
        \( f(z) = f(-z) \).
    \end{bsp}
    \pause
    \begin{bsp}
        Die Funktionenalgebra aller trigonometrischen Polynome auf \( [0,2\pi) \) 
        separiert Punkte und verschwindet nirgends.
    \end{bsp}
\end{frame}

\begin{frame}{Satz von Stone-Weierstraß}
    \begin{satz}[Satz von Stone-Weierstraß]
        Sei \(M\) ein kompakter metrischer Raum und \( \mathcal{A} \) eine 
        Funktionenalgebra in \( \mathcal{C}^0(M) \) die nirgends verschwindet und 
        Punkte unterscheidet. 
        Dann ist \( \mathcal{A} \) dicht in \( \mathcal{C}^0(M) \).
    \end{satz}
    \pause
    \begin{bem}
        Der Satz von Stone-Weierstraß ist offensichtlich eine Verallgemeinerung 
        des Satzes von Weierstraß. 
        Wir benötigen die Aussage des Satzes von Weierstraß für den Beweis von Stone-Weierstraß.
    \end{bem}
\end{frame}

\begin{frame}
    \begin{lem}
        Sei \( \A \subset \CM \) eine Funktionenalgebra, die nirgends verschwindet 
        und Punkte unterscheidet. 
        Dann existiert für beliebige \( p_1, p_2 \in M, p_1\neq q_2, 
        c_1, c_2 \in \R \)
        ein \(f \in \A\) mit 
        \[ f(p_1) = c_1, \quad f(p_2) = c_2. \]
    \end{lem}
\end{frame}

\begin{frame}{Beweis von Lemma 3}
    Seien \( p_1, p_2 \in M \) und \( c_1, c_2 \in \R \) gegeben. 
    \pause

    Da \(\A\) nirgends verschwindet, existieren \( g_1, g_2 \in \A \), 
    sodass 
    \( g_1(p_1) \neq 0, g_2(p_2) \neq 0 \). 
    \pause

    Also gilt \( g := g_1^2 + g_2^2 \in \A \). \(g\) verschwindet weder in \(p_1\) noch \(p_2\).
    \pause 

    \(\A\) unterscheidet Punkte, also existiert ein \( h \in \A \), sodass 
    \( h(p_1) \neq h(p_2) \).
    \pause 
    Betrachten wir die Matrix 

    \[ H = \begin{pmatrix}
        a & ab \\
        c & cd
    \end{pmatrix} = \begin{pmatrix}
        g(p_1) & g(p_1) h(p_1) \\
        g(p_2) & g(p_2) h(p_2)
    \end{pmatrix}. \]

    Nach Konstruktion gilt 
    \( a, c \neq 0 \) sowie \(b \neq d\). Also gilt \( \det H = acd - abc = ac(d - b) \neq 0 \). 
\end{frame}

\begin{frame}
    Somit hat das LGS 
    \begin{align*}
        a \xi + ab \eta &= c_1 \\
        c \xi + cd \eta &= c_2
    \end{align*}

    genau eine Lösung für \( \xi \) und \( \eta \). 
    \pause
    Sei \( f := \xi g + \eta g h \). Offensichtlich ist \( f \in \A \) und es gilt 
    \begin{align*}
        f(p_1) &= \xi g(p_1) + \eta g(p_1) h(p_1) = c_1, \\
        f(p_2) &= \xi g(p_2) + \eta g(p_2) h(p_2) = c_2.
    \end{align*}
    \qed
\end{frame}

\begin{frame}
    \begin{lem}
        Sei \( \A \subset \CM \) eine Funktionenalgebra 
        sowie \(M\) ein kompakter metrischer Raum.
        Dann ist \( \Abar \) ebenfalls eine Funktionenalgebra.
    \end{lem}
    Beweis: \pause 
    Stimmt so. \qed
\end{frame}

\begin{frame}
    \begin{lem}
        Sei \( \A \subset \CM \) eine Funktionenalgebra. Dann gilt 
        \[ f \in \Abar \Rightarrow \abs{f} \in \Abar \]
        sowie 
        \[ f_1, \ldots, f_n \in \Abar \Rightarrow \max(f_1,\ldots, f_n), \min(f_1,\ldots, f_n) \in \Abar. \]
    \end{lem} \pause
    Beweis:
    Nach dem Satz von Weierstraß existiert ein Polynom 
    \( p(y) \), sodass 
    \[ \sup \set{ \abs{ p(y) - \abs{y} } 
    \;|\; \abs{y} \leq \norm{f} } < \frac{\varepsilon}{2}, \]
    da \( \abs{y} \in \C([-\norm{f}, \norm{f}]) \). 
    \pause

    Der konstante Term dieses Polynoms ist höchstens \( \varepsilon / 2 \), 
    da \( \abs{ p(0) - \abs{0} } < \varepsilon/2 \).
\end{frame}

\begin{frame}
    Sei \( q := p - p(0) \). Dann ist \( q(y) \) ein 
    Polynom mit Konstante \(0\) und es gilt 
    \[ \abs{ q(y) - \abs{y} } < \varepsilon 
    \;\forall y \in [-\norm{f}, \norm{f}]. \]
    \pause

    Schreiben wir nun \( q \) als 
    \( q(y) = a_1 y + a_2 y^2 + \cdots + a_n y^n \) und 
    \[ g = a_1 f + a_2 f^2 + \cdots + a_n f^n. \]
    \pause

    Da \( \Abar \) eine Funktionenalgebra nach Lemma 3 (TODO)
    und somit gilt 
    \( g \in \Abar \). 
    \pause

    Sei \( x\in M \) und \( y = f(x) \). Dann gilt 
    \[ \abs{ g(x) - \abs{f(x)} } = \abs{ q(y) - \abs{y} } < \varepsilon. \]
    \pause
    
    Also folgt \( \abs{f} \in \Abarbar = \Abar \). 
\end{frame}
\end{frame}