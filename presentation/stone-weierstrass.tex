\section{Approximationssatz von Stone-Weierstraß}

\begin{frame}{Funktionenalgebra}
    \begin{defi*}{Funktionenalgebra}
        Eine Menge \( \mathcal{A} \subset \mathcal{C}^0(M) \) heißt 
        \textit{Funktionenalgebra}, wenn sie geschlossen unter Addition, 
        skalarer Multiplikation und Funktionenmultiplikation ist, 
        d. h. \( \forall f,g \in \A, c \in \mathbb{R} \)
        \[ f + g \in \mathcal{A}, \quad c \cdot f \in \mathcal{A}, \quad f \cdot g \in \mathcal{A}. \]
    \end{defi*}
\end{frame}

\begin{frame}{Funktionenalgebra}
    \begin{defi*}{Verschwinden}
        Eine Funktionenalgebra \(\A \subset \C0\) 
        \textit{verschwindet} an einem Punkt \(p \in M\), falls 
        \[ f(p) = 0 \;\forall f \in \mathcal{A}. \]
    \end{defi*}
    \begin{bsp}
        \( \A := \set{ x \mapsto \sum_{j=1}^n a_n x^n \;|\; n\in \N, a_j \in \R, j = 1,\ldots, n } \subset \mathcal{C}^0([-1,1]) \) 
        verschwindet im Punkt \(0\).
    \end{bsp}
\end{frame}

\begin{frame}{Punkte unterscheiden}
    \begin{defi*}
        Eine Funktionenalgebra \(\A \subset \C0\) \textit{unterscheidet Punkte}, 
        wenn \( \forall p_1, p_2 \in M \) eine Funktion \(f \in \A\) existiert, 
        sodass \( f(p_1) \neq f(p_2) \).
    \end{defi*}
    \begin{bsp}
        \( \A := \set{ x \mapsto \sum_{j=0}^n a_{j} x^{2j} \;|\; n\in\N, a_j \in \R, j = 1,\ldots, n } 
        \subset \mathcal{C}^0([-1,1]) \)
        separiert die Punkte \( z \) und \(-z\) (\(z \in [-1,1] \setminus \set{0} \)) nicht, da \( \forall f \in \A \) gilt 
        \( f(z) = f(-z) \).
    \end{bsp}
    \begin{bsp}
        Die Funktionenalgebra aller trigonometrischen Polynome auf \( [0,2\pi) \) 
        separiert Punkte und verschwindet nirgends.
    \end{bsp}
\end{frame}

\begin{frame}{Satz von Stone-Weierstrass}
    \begin{satz*}{Satz von Stone-Weierstrass}
        Sei \(M\) ein kompakter metrischer Raum und \( \mathcal{A} \) eine 
        Funktionenalgebra in \( \mathcal{C}^0(M) \) die nirgends verschwindet und 
        Punkte unterscheidet. 
        Dann ist \( \mathcal{A} \) dicht in \( \mathcal{C}^0(M) \).
    \end{satz*}

    \begin{bem}
        Der Satz von Stone-Weierstrass ist offensichtlich eine Verallgemeinerung 
        des Satzes von Weierstrass. 
        Wir benötigen die Aussage des Satzes von Weierstrass für den Beweis von Stone-Weierstrass.
    \end{bem}
\end{frame}